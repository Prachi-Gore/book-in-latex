\documentclass[12pt,onesided,a4paper]{book}
\usepackage[top=1.5cm, bottom=1.5cm, left=2cm, right=1cm]{geometry}
\usepackage{amsmath,amssymb,array,amsfonts,dsfont,multirow}
\usepackage{adjustbox}
\usepackage{color,paralist,graphicx,wrapfig,tikz}
\usepackage[mathscr]{euscript}
\usepackage{titlesec}
\usepackage{fancyhdr}%,fancyvbr}
\usepackage[T1]{fontenc}
\setcounter{tocdepth}{1}
\usepackage{hyperref}
\hypersetup{
%    bookmarks=true,         % show bookmarks bar?
%    unicode=false,          % non-Latin characters in Acrobat’s bookmarks
%    pdftoolbar=true,        % show Acrobat’s toolbar?
%    pdfmenubar=true,        % show Acrobat’s menu?
%    pdffitwindow=false,     % window fit to page when opened
%    pdfstartview={FitH},    % fits the width of the page to the window
%    pdftitle={My title},    % title
%    pdfauthor={Author},     % author
%    pdfsubject={Subject},   % subject of the document
%    pdfcreator={Creator},   % creator of the document
%    pdfproducer={Producer}, % producer of the document
%    pdfkeywords={keyword1, key2, key3}, % list of keywords
%    pdfnewwindow=true,      % links in new PDF window
    colorlinks=true,       % false: boxed links; true: colored links
    linkcolor=black,          % color of internal links (change box color with linkbordercolor)
%    citecolor=green,        % color of links to bibliography
%    filecolor=magenta,      % color of file links
%    urlcolor=cyan           % color of external links
}

\usepackage{float}
\usepackage{tcolorbox,amsmath}
\usepackage{lipsum}
\tcbuselibrary{skins,breakable}
\usetikzlibrary{shadings,shadows}
%\usepackage[dvipsnames]{xcolor}

\newenvironment{myexampleblock}[1]{%
    \tcolorbox[beamer,%
    noparskip,breakable,
    colback=LightGreen,colframe=DarkGreen,%
    colbacklower=LimeGreen!75!LightGreen,%
    title=#1]}%
    {\endtcolorbox}

\newenvironment{myalertblock}[1]{%
    \tcolorbox[beamer,%
    noparskip,breakable,
    colback=LightCoral,colframe=DarkRed,%
    colbacklower=Tomato!75!LightCoral,%
    title=#1]}%
    {\endtcolorbox}

\newenvironment{myblock}[1]{%
    \tcolorbox[beamer,%
    noparskip,breakable,
    colback=LightBlue,colframe=DarkBlue,%
    colbacklower=DarkBlue!75!LightBlue,%
    title=#1]}%
    {\endtcolorbox}

\renewcommand{\rmdefault}{ptm}
\newcommand{\ds}{\displaystyle}
\newcommand{\rs}{\includegraphics[height=9pt,width=5.5pt]{rupee.pdf}\text{ }}
\usepackage[scaled=0.92]{helvet}
\usepackage{fourier}
\newcommand{\qed}{\nobreak \ifvmode \relax \else
 \ifdim\lastskip<1.5em \hskip-\lastskip
 \hskip1.5em plus0em minus0.5em \fi \nobreak
 \vrule height0.75em width0.5em depth0.25em\fi}
\usepackage{fancyhdr}
\author{Prof. R. L. Shinde}
\title{Actuarial Statistics}
\renewcommand{\footrulewidth}{0.4pt}
\pagestyle{fancy} \setcounter{page}{1}
%\renewcommand\chaptermark[1]{\markboth{\thechapter\ #1}{}}
%\fancyhead[ro,le]{\footnotesize\textit{May 17,2019}}
\fancyhead[ro,re]{\footnotesize \textit{\nouppercase{\leftmark}}}
\fancyfoot[lo,le]{\footnotesize\thepage}
\fancyhead[lo,le]{\footnotesize\textit{M.Sc.(Statistics) Lecture Notes (Prof. R. L. Shinde)}}
\fancyfoot[ro,re]{\footnotesize\textit{Department of Statistics, Kavayitri Bahinabai Chaudhari North Maharashtra University, Jalgaon}}
\cfoot{}
\raggedbottom
\usepackage[T1]{fontenc}
\usepackage{titlesec, blindtext, color}
\definecolor{gray75}{gray}{0.75}
\newcommand{\hsp}{\hspace{20pt}}
\newcommand{\twocol}[5]{\item #1 \par\nopagebreak[4]\medskip \begin{tabular}{lp{7cm}lp{7cm}}
 $a$)& #2 & $b$) &#3 \\
 $c$)& #4 & $d$) &#5 \\
\end{tabular}}
\usepackage{tcolorbox}
%\tcbuselibrary{breakable}
%\newcolorbox{mybox}{colback=red,colframe=black}
\usepackage{fourier}
\usepackage[Lenny]{fncychap}
\usepackage{amsthm}
\usepackage{amssymb}
\usepackage{actuarialsymbol}
\usepackage{multirow}
\newtheorem{theorem}{Theorem}[section]
\newtheorem{corollary}{Corollary}[theorem]
\newtheorem{lemma}[theorem]{Lemma}

\newcounter{examplecounter}
\newenvironment{example}{
    \refstepcounter{examplecounter}%
  \noindent\textbf{Example \arabic{examplecounter}:}%
  \quad
}{%
}

\newenvironment{solution}{
  \noindent\textbf{Solution:\newline}%
  \quad
}{%
\hfill\ensuremath{\blacksquare}
}

\newtheoremstyle{mystyle}%                % Name
  {}%                                     % Space above
  {}%                                     % Space below
  {\itshape}%                             % Body font
  {}%                                     % Indent amount
  {\bfseries}%                    % Theorem head font
  {: }%                                    % Punctuation after theorem head
  { }%                                    % Space after theorem head, ' ', or \newline
  {}%                                     % Theorem head spec (can be left empty, meaning `normal')

\theoremstyle{mystyle}
\newtheorem{definition}{Definition}

\begin{document}
\large
\maketitle
% \tableofcontents

% \input{2101}
%\input{2102}
%\input{2104}
%\input{2105}
%\input{2106}
%\input{2108}
%\input{2109}
%\input{2110}
%\input{2111}
%\input{2112}
% \chapter{}


$$P[x<X<x+n]=S(x)-S(x+n)$$
Hence we get,
$$\dx[n]{x}=E[\Dx[n]{x}]=l_0[S(x)-S(x+n)]$$
$$n{d_x}=l_x-l_{x+n}$$
when n=1,we omit the prefixes on $\Dx[n]{x}$ and $\dx[n]{x}$ \\
$\Dx[1]{x}$ is equivalent to $\Dx[]{x}$  and $\dx[1]{x}$ is equivalent to $\dx[]{x}$


\section{Relation between $u_x$ and $l_x$}
we have,$$l_x=l_0S(x)$$
$$log(l_x)=log(l_0)+log(S(x))$$
note that as S(x) is differentiable function of x and hence $l_x$ is also differentiable function of x\\
\begin{align*}
\frac{-d}{dx} log(l_x)=\frac{-d}{dx} log(S(x)\\
\frac{-1}{l_x}\frac{d}{dx} l_x=\frac{-1}{S(x)}\frac{d}{dx} S(x)=\frac{-S'(x)}{S(x)}=u_x
\end{align*}
\begin{align*}
\frac{-d}{dx} l_x=u_xl_x\\
-dl_x=l_xu_xd_x
\end{align*}
the factor $l_xu_x$ can be interpreted as the expected density of deaths in the age interval (x,x+$d_x$)\\
\begin{align*}
l_x=l_0S(x)&=l_0e^{-\int_{0}^{x} u_y dy}\\
l_{x+n}&=l_x \px[n]{x}\\
l_{x+n}&=l_xe^{-\int_{x}^{x+n} u_y dy}
\end{align*}

Also as,
\begin{align*}
\frac{-d}{dy} l_y&=u_yl_y\\
\int_{x}^{x+n}\left(\frac{-d}{dy}l_y\right) dy&=\int_{x}^{x+n}l_y u_y dy\\
l_x-l_{x+n}&=\int_{x}^{x+n}l_y u_y dy\\
\dx[n]{x}=l_x-l_{x+n}&=\int_{x}^{x+n}l_y u_y dy
\end{align*}
for convenience of reference,we call this concept of $l_0$ newborns,each with survival function S(x) a random survivorship group\\
In the following we have an example of life table.


\begin{table}[!h]
\caption{Life Table for the total population:US(1979-81)}
\centering
\begin{tabular}{|l|l|l|l|l|l|l|}
\hline
Age Interval                                  & $\qx[t]{x}$                                           & $l_x$                                            & $\dx[t]{x}$                                           & $\Lx[t]{x}$                                           & $T_x$                                            & E(T(x))                                       \\ \hline
days                                          &                                               &                                               &                                               &                                               &                                               &                                               \\ \hline
0-1                                           & 0.00463                                       & 100,000                                       & 463                                           & 273                                           & 7387758                                       & 73.88                                         \\ \hline
1-7                                           & 0.00246                                       & 99537                                         & 245                                           & 1635                                          & 7387485                                       & 74.22                                         \\ \hline
7-28                                          & 0.00139                                       & 99292                                         & 138                                           & 5708                                          & 7385850                                       & 74.38                                         \\ \hline
28-365                                        & 0.00418                                       & 99154                                         & 414                                           & 91357                                         & 7380142                                       & 74.43                                         \\ \hline
years                                         &                                               &                                               &                                               &                                               &                                               &                                               \\ \hline
0-1                                           & 0.01260                                       & 100000                                        & 1260                                          & 98973                                         & 7387758                                       & 73.88                                         \\ \hline
1-2                                           & 0.00093                                       & 98740                                         & 92                                            & 98694                                         & 7288785                                       & 73.82                                         \\ \hline
2-3                                           & 0.00065                                       & 98648                                         & 64                                            & 98617                                         & 7190091                                       & 72.89                                         \\ \hline
3-4                                           & 0.00050                                       & 98584                                         & 49                                            & 98560                                         & 7091474                                       & 71.93                                         \\ \hline
\begin{tabular}[c]{@{}l@{}}:\\ :\end{tabular} & \begin{tabular}[c]{@{}l@{}}:\\ :\end{tabular} & \begin{tabular}[c]{@{}l@{}}:\\ :\end{tabular} & \begin{tabular}[c]{@{}l@{}}:\\ :\end{tabular} & \begin{tabular}[c]{@{}l@{}}:\\ :\end{tabular} & \begin{tabular}[c]{@{}l@{}}:\\ :\end{tabular} & \begin{tabular}[c]{@{}l@{}}:\\ :\end{tabular} \\ \hline
108-109                                       & 0.35453                                       & 51                                            & 18                                            & 42                                            & 115                                           & 2.24                                          \\ \hline
109-110                                       & 0.35988                                       & 33                                            & 12                                            & 27                                            & 73                                            & 2.2                                           \\ \hline
\end{tabular}
\end{table}

% Please add the following required packages to your document preamble:
% \usepackage{multirow}
\begin{table}[!h]
\caption{Following table provides values of $q_x$ from some National Life tables}
\centering
\begin{tabular}{|c|cc|cc|cc|}
\hline
\multirow{2}{*}{\begin{tabular}[c]{@{}c@{}}x\\ (years)\end{tabular}} & \multicolumn{2}{c|}{\begin{tabular}[c]{@{}c@{}}Australian\\ 2000-02\end{tabular}} & \multicolumn{2}{c|}{\begin{tabular}[c]{@{}c@{}}English\\ 1990-92\end{tabular}} & \multicolumn{2}{c|}{\begin{tabular}[c]{@{}c@{}}US\\ 2002-04\end{tabular}} \\ \cline{2-7} 
                                                                     & \multicolumn{1}{c|}{Male}                          & Female                       & \multicolumn{1}{c|}{Male}                        & Female                      & \multicolumn{1}{c|}{Male}                      & Female                   \\ \hline
0                                                                    & \multicolumn{1}{c|}{0.00567}                       & 0.00466                      & \multicolumn{1}{c|}{0.00814}                     & 0.00632                     & \multicolumn{1}{c|}{0.00764}                   & 0.00627                  \\ \hline
1                                                                    & \multicolumn{1}{c|}{0.00044}                       & 0.00043                      & \multicolumn{1}{c|}{0.00062}                     & 0.00055                     & \multicolumn{1}{c|}{0.00053}                   & 0.00042                  \\ \hline
2                                                                    & \multicolumn{1}{c|}{0.00031}                       & 0.00019                      & \multicolumn{1}{c|}{0.00038}                     & 0.00030                     & \multicolumn{1}{c|}{0.00037}                   & 0.00028                  \\ \hline
10                                                                   & \multicolumn{1}{c|}{0.00013}                       & 0.00008                      & \multicolumn{1}{c|}{0.00018}                     & 0.00013                     & \multicolumn{1}{c|}{0.00018}                   & 0.00013                  \\ \hline
20                                                                   & \multicolumn{1}{c|}{0.00096}                       & 0.00036                      & \multicolumn{1}{c|}{0.00084}                     & 0.00031                     & \multicolumn{1}{c|}{0.00139}                   & 0.00045                  \\ \hline
40                                                                   & \multicolumn{1}{c|}{0.00159}                       & 0.00088                      & \multicolumn{1}{c|}{0.00172}                     & 0.00107                     & \multicolumn{1}{c|}{0.00266}                   & 0.00149                  \\ \hline
90                                                                   & \multicolumn{1}{c|}{0.15934}                       & 0.12579                      & \multicolumn{1}{c|}{0.20465}                     & 0.15550                     & \multicolumn{1}{c|}{0.16805}                   & 0.13328                  \\ \hline
100                                                                  & \multicolumn{1}{c|}{0.24479}                       & 0.23863                      & \multicolumn{1}{c|}{0.38705}                     & 0.32489                     & \multicolumn{1}{c|}{-}                         & -                        \\ \hline
\end{tabular}
\end{table}








    




% \chapter{}

The Actuarial present value for this insurance is,
$$(DA)|_{x:\angln}=\sum_{k=0}^{n-1}(n-k)v^{k+1}\px[k]{x}q_{x+k}$$
consider,
$$(DA)|_{x:\angln}=\sum_{k=0}^{n-1}(n-k)v^{k+1}\px[k]{x}q_{x+k}$$\\
$$=\sum_{k=0}^{n-1}(n-k)_{k|}\Ax{^|{x:\angl1}}$$\\
as by defination of actuarial present value of a unit benefit in an m-year deferred n year term insurance

$$_{m|}{\Ax{^|{x:\angln}}}=\sum_{k=m}^{m+n-1}v^{k+1}\px[k]{x}q_{x+k} $$
with m=k and n=1 , we get $_{k|}\Ax{^|{x:\angl1}}=v^{k+1}\px[k]{x}q_{x+k}$
so we have,
$$(DA)|_{x:\angln}=\sum_{k=0}^{n-1}(n-k)_{k|}\Ax{^|{x:\angl1}}=\sum_{k=0}^{n-1}(n-1)_{k|}\Ax{x}$$
Also in above expression if we substitute,
$$n-k=\sum_{j=0}^{n-k-1}(1)$$
we get,
$$(DA)|_{x:\angln}=\sum_{k=0}^{n-1}\sum_{j=0}^{n-k-1}(1)v^{k+1}\px[k]{x}q_{x+k}$$
By interchanging the order of summation we obtain ,
$$\sum_{k=0}^{n-1}\sum_{j=0}^{n-k-1}(1)v^{k+1}\px[k]{x}q_{x+k}=\sum_{j=0}^{n-1}\Ax{^|{x:\angl{n-j}}}=\Ax{^|{x:\angl{n-j}}}$$
This is Actuarial present value of  $(n-j)$ year term insurance  when benefit of 1  unit is payable at the end of death year.\\
Hence we obtain,one more expression for $(DA)|_{x:\angln}$ as,
$$(DA)|_{x:\angln}=\sum_{j=0}^{n-1}\Ax{^|{x:\angl{n-j}}}$$
In the following we give the summary of different insurances under which benefit amount is payable at the end of year of death.we also provide the summary of the recursion relations for their actuarial present values(NSPs)

% 3rd page
% https://www.tablesgenerator.com/latex_tables#

\begin{table}[H]
\caption{Summary of insurances payable at the end of year of death}
\begin{tabular}{|l|l|l|l|l|}
\hline
(1)                                                                             & (2)                                                                                            & (3)                                                                                                                         & (4)                                                                                                                                       & (5)                                                                                          \\ \hline
\multicolumn{1}{|c|}{\begin{tabular}[c]{@{}c@{}}Insurance \\ Name\end{tabular}} & \multicolumn{1}{c|}{\begin{tabular}[c]{@{}c@{}}benefit function\\ $b_{k+1}$\end{tabular}}    & \multicolumn{1}{c|}{\begin{tabular}[c]{@{}c@{}}Discount function\\ $v_{k+1}$\end{tabular}}                                 & \multicolumn{1}{c|}{\begin{tabular}[c]{@{}c@{}}present value function\\ $z_{k+1}$\end{tabular}}                                          & {\begin{tabular}[c]{@{}c@{}}APV\end{tabular}} \\ \hline
whole life                                                                      & 1                                                                                              & $v^{k+1}$                                                                                                  & $v^{k+1}$                                                                                                                & $A_x$                           \hspace{0.5cm} (*)                                                          \\ \hline
n year term                                                                     & \begin{tabular}[c]{@{}l@{}}1\hspace{0.7cm} k=0,1,...,n-1\\ 0\hspace{0.8cm}k=n,n+1,...\end{tabular}                        & $v^{k+1}$                                                                                                  & \begin{tabular}[c]{@{}l@{}}$v^{k+1}$\hspace{1cm} k=0,1,...,n-1\\ 0\hspace{1.6cm} k=n,n+1,...\end{tabular}                              & $\Ax{^|{x:\angln}}$\hspace{0.2cm}  (*)                \\ \hline
n year endowment                                                                & 1                                                                                              & \begin{tabular}[c]{@{}l@{}}$v^{k+1}$\hspace{0.25cm} k=0,1,...,n-1\\ $v^{n}$ \hspace{0.6cm}k=n,n+1,...\end{tabular} & \begin{tabular}[c]{@{}l@{}}$v^{k+1}$ \hspace{1cm}k=0,1,...,n-1\\ $v^{n}$ \hspace{1.3cm}    k=n,n+1,...\end{tabular}               & $\Ax{x:\angln}$\hspace{0.3cm} (*)                                 \\ \hline
\begin{tabular}[c]{@{}l@{}}m year deferred \\ n year term\end{tabular}          & \begin{tabular}[c]{@{}l@{}}1\hspace{0.8cm}k=m...m+n-1\\ 0\hspace{0.8cm}k=0,...m-1\\\hspace{1cm}k=m+n,m+n+1,...\end{tabular} & $v^{k+1}$                                                                                                  & \begin{tabular}[c]{@{}l@{}}$v^{k+1}$\hspace{1cm} k=m...m+n-1\\ 0 \hspace{1.65cm}k=0,...m-1\\                \hspace{1.95cm}k=m+n,...\end{tabular} & $\Ax[m|n]{x}$ \hspace{0.15cm}(*)              \\ \hline
\begin{tabular}[c]{@{}l@{}}n year term\\ increasing annuity\end{tabular}        & \begin{tabular}[c]{@{}l@{}}k+1\hspace{0.3cm} k=0,1,...n-1\\ 0 \hspace{0.7cm}     k=n,n+1,...\end{tabular}                  & $v^{k+1}$                                                                                                  & \begin{tabular}[c]{@{}l@{}}(k+1)$v^{k+1}$\hspace{0.25cm}k=0,1,...n-1\\ 0\hspace{1.73cm}k=n,n+1,...\end{tabular}                 & $(IA)|_{x:\angln}$                                                           \\ \hline
\begin{tabular}[c]{@{}l@{}}n year term \\decreasing annuity\end{tabular}        & \begin{tabular}[c]{@{}l@{}}n-k \hspace{0.3cm} k=0,1,...n-1\\ 0 \hspace{0.7cm}  k=n,n+1,...\end{tabular}                    & $v^{k+1}$                                                                                                  & \begin{tabular}[c]{@{}l@{}}(n-k)$v^{k+1}$\hspace{0.25cm} k=0,1,...n-1\\ 0\hspace{1.7cm}                     k=n,n+1,..\end{tabular}                   & $(DA)|_{x:\angln}$                                          \\ \hline
\begin{tabular}[c]{@{}l@{}}whole life\\ increasing annuity\end{tabular}         & k+1 \hspace{0.4cm}k=0,1,...                                                                                  & $v^{k+1}$                                                                                                  & (k+1)$v^{k+1}$,\hspace{0.3cm}k=0,1,...                                                                                                 & $(IA)_x$                                                                                      \\ \hline
\end{tabular}
\begin{center}
    {(*):Rule of moments holds,thus, v(Z)=2A-$A^2$}
\end{center}

\end{table}



























    




%\input{2114}
%\input{2115}
%\input{2116}
%\input{2117}
%\input{2118}
%\input{2119}
%\input{2120}
%\input{2121}
%\input{2122}
%\input{2123}
%\input{2124}
%\input{2125}
%\input{2126}
%\input{2127}
%\input{2128}
% \input{2129}
%\input{2130}
%\input{2131}
%\input{2132}
%\input{2133}
%\input{2134}
%\input{2136}
%\input{2137}
%\input{2138}
%\input{2139}
%\input{2140}
%\input{2141}
%\input{2143}
%\input{2144}
%\input{2145}
%\input{2146}
%\input{2147}
%\input{2149}

\end{document}
